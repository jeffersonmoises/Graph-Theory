\quest{4.2.8}

($\Leftarrow$) Sabemos que um digrafo é bipartido se e somente se todas suas
componentes fortes são bipartidas. Portanto se $D$ é um digrafo que contém uma
componente forte não bipartida, então $D$ não é bipartido. Pelo Teorema 4.7, $D$
não contém um ciclo ímpar. Isso significa que nenhuma das componentes fortes de
$D$ possui um ciclo impar. Portanto, pelo Exercício 3.4.11b, $D$ não possui ciclo
ímpar orientado.

($\Rightarrow$) Se um digrafo $D$ contém um ciclo orientado ímpar é porque alguma
de suas componentes fortes contém um ciclo orientado ímpar. Então, pelo Exercício
3.4.11b, $D$ possui um ciclo ímpar e do Teorema 4.7, $D$ não é bipartido. Mas se
$D$ não é bipartido então existe uma componente forte de $D$ tal que esta não é
bipartida.

