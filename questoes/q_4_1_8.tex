\quest{4.1.8}

\partQuest{a} Sejam $T$ e $T'$ como na hipótese, $D(v) = \max\{d(v,u) \, : u \in V\}$
tal que $D(v)$ é mínimo e $C = \{v_1, v_2, \cdots, v_k\}$ os centros de $T$.

\begin{fact}
\label{fact:folhadist}
Se $D(v) = d(v, f)$ então $f$ é folha.
\end{fact}

\begin{proof} Suponha por absurdo $D(v)$ como na hipótese mas que $f$ não seja
folha. Porque $f$ não é folha, $f$ possui um vizinho $u \ne v$. Nesse caso,
$D'(v) = d(v, f) + 1 > D(v)$, o que é um absurdo.
\end{proof}

\begin{fact}
\label{fact:contained}
Em $T'$, $D_{T'}(u) = D_T(u) - 1$, $\forall u \in V(T')$.
\end{fact}

\begin{proof} Direta do Fato \ref{fact:folhadist} e da forma como $T'$ é obtida.
\end{proof}

Seja $C' = \{v'_1, v'_2, \cdots, v'_l\}$ os centros de $T'$. Do Fato
\ref{fact:contained}, obtemos que $C \subseteq C'$. Suponha agora por absurdo que
$\exists v'_i \in C' \setminus C$ tal que $D_{T'}(v'_i) \le D_{T'}(v)$, $v \in C$.
Então sabemos do Fato \ref{fact:contained} que
\begin{equation}
\label{eq:part1}
D_{T'}(v'_i) \le D_{T'}(v) = D_T(v) - 1
\end{equation}

Além do mais, sabemos que
\begin{equation}
\label{eq:part2}
D_{T'}(v'_i) \le D_T(v'_i) \le D_{T'}(v'_i) + 1
\end{equation}

Finalmente, temos de \eqref{eq:part1} e \eqref{eq:part2} que
\[D_T(v'_i) = D_{T'}(v'_i) + 1 \le D_T(v),\] o que é um absurdo pois nesse caso
$v'_i$ seria centro em $T$.

\partQuest{b}

