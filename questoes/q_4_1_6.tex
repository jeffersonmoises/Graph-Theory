\quest{4.1.6}

Seja $x \in V(D)$ e $X=\{v: v \in v(D)$ e $\exists xDv\}$.
%
Considere então o subgrafo induzido $D'$ pelos vértices em $X$ ($D' := D[X]$).
%
Remova todos os arcos com cabeça em $x$, pois queremos um $x$-{\it branching}.
%
Todos os vértices ainda permanecem alcançáveis, pois estes arcos não estavam em nenhum caminho orientado entre $x$ e $v$, $v \in V(D')$ (estavam em passeios orientados).
%
Se $\forall v \in V(D')\quad d^-(v) = 1$, então já temos um $x$-{\it branching}.
%
Caso contrário, seja $v \in V(D')$, onde $d^-(v) > 1$, então remova um dos arcos com cabeça em $v$.
%
Neste caso, $v$ ainda será alcançável a partir de $x$, pois ainda há um caminho $xD'v$ utilizando um dos arcos $(v',v)$ restantes, pois $\exists xD'v'$.
%
Desta forma, remover um arco de vértices $v$, onde $d^-(v) > 1$, preserva a existência de caminho entre $x$ e $v$.
%
Portanto, repetindo este processo, eventualmente obteremos um $x$-{\it branching}, já que $e(D')$ é finito.
\fimprova
