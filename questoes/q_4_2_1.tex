\quest{4.2.1}

\partQuest{a} Seja $T_1$ o conjunto de árvores geradoras de $G$ que contém $e$ e
$T_2$ o conjunto de árvores geradoras de $G/e$. Seja $f: T_1 \to T_2$ um função
bijetora  que associa uma árvore $T \in T_1$ com $T' \in T_2$. Para isso $f$
contrai a aresta $e$ de $T$ e associa a subárvore resultante $T^* = T/e$ com a
subárvore $T \in T_2$ tal que $T^* \cong T$. Reciprocamente, $f^{-1}: T_2 \to
T_1$ subdivide a aresta $e$ de $T \in T_2$ e associa a subárvore resultante $T^*$ com a subárvore $T' \in T_1$ tal que $T^* \cong T'$.\\

\partQuest{b} Seja $G$ um grafo conexo qualquer e $e = vx \in E(G)$, temos que para cada subárvore geradora $T$ de $G$, ou $e \in E(T)$ ou $e \notin E(T)$.
%
Se $e \notin E(T)$, então $T$ é arvore geradora de $G\backslash e$, caso contrário, se $e \in E(T)$, então pela letra (a) temos que $T/e$ é árvore geradora de $G/e$.
%
Logo, $$t(G) = t(G\backslash e) + t(G/e)$$
