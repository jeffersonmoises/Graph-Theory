\quest{5.2.2}

\partQuest{a} Suponha $G$ e $e$ como na hipótese. Se $G\setminus e$ é não
separável, sabemos do Teorema 5.2 que quaisquer duas arestas de $G\setminus e$
pertencem a um ciclo em comum. A adição de $e$ não pode desfazer essa
propriedade. Resta agora mostrar que $e$ e $f$, $\forall f \in E(G\setminus
e)$, pertencem a um ciclo em comum.  Porque $G\setminus e$ é não separável,
então $G$ é conexo e como $e$ não é \emph{loop}, então $e$ é corda em $G$.
Portanto, $G$ é não separável.
\fimprova

\partQuest{b} Primeiramente, considere que $v(G/e) = 1$, então  $G$ é um $P_2$
ou $C_2$ e, em ambos os casos, é não separável. Por outro lado, considere que
$v(G/e) \ge 2$. Denote $e = xy \in E(G)$ e $v_{xy}$ o vértice correspondente
em $V(G/e)$, então $\forall a,b \in V(G/e)$ há $P$ e $P'$ dois $a,b$-caminhos
internamente disjuntos em $G/e$, já que este não é separável. Logo, como no 
máximo um destes caminhos ($P$ ou $P'$) pode usar $v_{xy}$ (spg. $P$), então
podemos construir dois caminhos $P'$ e $P'' = aPxyPb$, internamente disjuntos
em $G$ e, portanto, $G$ é não separável.
\fimprova
