\quest{4.1.20}

\partQuest{a}
($\Rightarrow$) Suponha por absurdo que $T_1 \cap T_2 = \varnothing$.
%
Se $u \in V(T_1)$ e $v \in V(T_2)$ então não existe caminho de $u$ para $v$ em $T_1 \cup T_2$, pois não há vertices em comum entre $T_1$ e $T_2$.
%
Mas isso contraria a hipótese de que $T_1 \cup T_2$ é uma subárvore de $T$.
%
Logo, $T_1 \cap T_2 \ne \varnothing$.

($\Leftarrow$) Seja $T_1$ e $T_2$ subárvores de $T$, tal que $T_1 \cap T_2 \ne \varnothing$. 
%
Obviamente $T_1 \cap T_2$ é acíclico, pois senão ambas árvores teriam ciclos e, além disto, $T_1 \cap T_2$ é conexo, pois se $u,v \in V(T_1 \cap T_2)$, então $u,v \in V(T_1)$ e, por definição $\exists uT_1v$.
%
Temos também que $T_1 \cup T_2$ é um grafo acíclico, senão $T$ conteria um ciclo, o que é um absurdo pois $T$ é árvore.
%
Resta-nos mostrar que $T_1 \cup T_2$ é conexo, e,  para tanto, considere $u,v \in V(T_1 \cup T_2)$.
%
Se $u,v \in V(T_1)$ ou $u,v \in V(T_2)$, temos, por definição, que existe caminho entre $u$ e $v$.
%
Sem perda de generalidade, considere que $u \in V(T_1)$ e $v \in V(T_2)$.
%
Seja $y \in T_1 \cap T_2$, então $\exists uT_1y$ e $\exists yT_2v$, logo, $\exists uT_1yT_2v$ em $T_1 \cup T_2$.
\fimprova

\partQuest{b}
Prova por indução forte em $|{\cal T}| = n$.

Hipótese: Se ${\cal T}$ é uma família de subárvores de uma árvore $T$ com
$|{\cal T}| = k$, $ 1 \le k \le n$, se quaisquer 2 membros de ${\cal T}$ possuem
um vértice em comum, então há um vértice de $T$ que pertence a todos os membros
de ${\cal T}$.

Passo: Seja ${\cal T}$ uma família de subárvores da árvore $T$ como na hipótese e
tal que $|{\cal T}| = n + 1$. Sejam $T'$ e $T''$ subárvores de ${\cal T}$ tal que
$T' \cap T'' \ne \varnothing$. Sabemos de (a) que $T' \cap T''$ é subárvore de $T$.
Seja ${\cal T'} = {\cal T} - \{T', T''\} \cup \{T' \cap T''\}$ a família obtida
por substituir $T'$ e $T''$ por sua intersecção em ${\cal T}$. Em ${\cal T'}$,
temos que $T_i \cap T_j \ne \varnothing$, $\forall T_i, T_j$. Então, por H.I.
existe um vértice $v$ de $T$ que pertence a todos os membros de ${\cal T'}$. Em
particular, $v$ pertence a $T' \cap T''$ e, portanto, pertence a $T'$ e $T''$.
\fimprova
