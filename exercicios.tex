\documentclass[10pt]{article}

\usepackage[utf8]{inputenc}
\usepackage[portuguese]{babel}
\usepackage{auto-pst-pdf}
\usepackage{fullpage}
\usepackage{latexsym}
\usepackage{vaucanson-g}
\usepackage{amsmath}
\usepackage{amssymb}

\newcommand{\quest}[1] {\vspace{0.5cm}\noindent {\bf {#1})}\\}
\newcommand{\partQuest}[1] {\noindent {\bf {#1})}}
\newcommand{\fimprova}{\begin{flushright}$\Box$\end{flushright}}

%%%%%%%%%%%%%%%%%%%%%%%%%%%%%%%%%%%%%%%%%%%%%%%%%%%%%
%
%
%			Título
%
%
%%%%%%%%%%%%%%%%%%%%%%%%%%%%%%%%%%%%%%%%%%%%%%%%%%%%%
\title{ {\footnotesize
	\hrule\vspace{1pt}\hrule\vspace{1ex}
		Instituto de Computação \hfill Universidade Estadual de Campinas
	\smallskip 
	\hrule\vspace{1pt}\hrule}\vspace{10pt}
		MO405 --- Teoria de Grafos I \\[-6pt]
	\author{Jefferson e Rafael} 
}

\date{\bf Primeiro Semestre de 2011}

%%%%%%%%%%%%%%%%%%%%%%%%%%%%%%%%%%%%%%%%%%%%%%%%%%%%%
%
%
%			Soluçőes
%
%
%%%%%%%%%%%%%%%%%%%%%%%%%%%%%%%%%%%%%%%%%%%%%%%%%%%%%
\begin{document}
 
\maketitle
\vspace{0.5cm}
\thispagestyle{empty}


% Questőes
% ==================== PRIMEIRA LISTA =====================
%\quest{1.1.2}

Seja $G$ um grafo simples com $n$ vértices numerados de $1$ até $n$.
%
Considere a matriz de adjacências {\bf A}$_{n,n}$ associada a $G$.
%
Neste caso a matriz é definida como $${\bf A}_{i,j} = \left\{ \begin{array}{lr} 1, & se\quad\{i,j\} \in E(G) \\ 0, & c. c. \end{array}\right.$$
%
Como $G$ é simples $A_{i,i}$, $1 \le i \le n$ e, portanto, $\sum_{1 \le i \neq j \le n} A_{i_j} \le n\cdot (n-1)$.
%
Pelo teorema fundamental temos que 
\begin{eqnarray}
	2m &=& \sum_{1 \le i \neq j \le n} A_{i_j} \le n\cdot(n-1) \nonumber\\
	m  &\le& \frac{n\cdot(n-1)}{2} = \left(\begin{array}{c} n \\ 2 \end{array}\right) \nonumber
\end{eqnarray}
\fimprova

Além disto, quando $G$ é completo ($K_n$) temos que a matriz {\bf A} fica definida como $${\bf A}_{i,j} = \left\{ \begin{array}{lr} 1, & se\quad ssi\neq j \\ 0, & c. c. \end{array}\right.$$ logo
\begin{eqnarray}
	2m &=& \sum_{1 \le i \neq j \le n} A_{i_j} = n\cdot(n-1) \nonumber\\
	m  &=& \frac{n\cdot(n-1)}{2} = \left(\begin{array}{c} n \\ 2 \end{array}\right) \nonumber
\end{eqnarray}
\fimprova

%\quest{1.1.3}

\partQuest{a} Seja $G[X,Y]$ um grafo simples bipartido, onde $|X| = r$ e $|Y| = s$, considere a matriz de adjac�ncias {\bf A}$_{r,s}$ associada a $G$.
%
Neste caso a matriz � definida como $${\bf A}_{i,j} = \left\{ \begin{array}{lr} 1, & se\quad\{i,j\} \in E(G) \\ 0, & c. c. \end{array}\right.$$
%
Portanto, como n�o h� v�rtices repitidos nas linhas e colunas da matriz {\bf A} temos que $$m = \sum_{\substack{1 \le i \le r \\ 1 \le j \le s}} A_{ij} \le rs$$.
\fimprova

\partQuest{b} Basta mostrar o m�ximo da fun��o $f(r,s) = r.s$, onde $1 \le r \le n$, $1 \le s \le n$ e $r = s$.
%
Para isto buscamos os pontos cr�ticos de $f$ com: 
\begin{eqnarray}
	\frac{\partial f}{\partial r} = 0 &\Rightarrow& s = 0 \nonumber\\
	\frac{\partial f}{\partial s} = 0 &\Rightarrow& r = 0 \nonumber
\end{eqnarray}
Como nenhum dos pontos pertence ao dom�nio, devemos analisar os pontos extremais nos limites do dom�nio, em particular, quando $r = s$.
%
Neste caso, $f(r,s) = r(n - r) = nr - r^2$, onde $n$ � constante e $1 \le r \le n$.
%
Buscando os pontos de m�nimo temos: $$\frac{\partial f}{\partial r} = 0 \Rightarrow n - 2r = 0 \Rightarrow r = \frac{n}{2}$$

Al�m disto, � $n = \frac{r}{2}$ � ponto de m�ximo pois: $$\frac{\partial f^2}{\partial r\partial r} = -2 < 0$$.

Portanto, o m�ximo da fun��o $f$ ocorre quando $r = s = \frac{n}{2}$ com valor $\frac{n^2}{4}$.
\fimprova

\partQuest{b} Os grafos bipartidos completos $K_{\frac{n}{2},\frac{n}{2}}$ possuem $n$ v�rtices e, exatamente $\frac{n^2}{4}$ arestas.



% ==================== SEGUNDA LISTA ======================
\quest{4.1.1}

\partQuest{a} {\bf Sol 1)} Provaremos primeiro um resultado mais genérico.
%
Seja $F$ uma floresta com $\Delta(F) = k$, então $F$ tem pelo menos $k$ folhas (Denote por $\# folhas(G)$ o número de folhas na árvore $G$). A prova será por indução forte em $e(F)$.

Se $e(F) = 0$, então o resultado segue trivialmente pois $\# folhas(F) \ge 0$.
%
Caso contrário, seja $P = v_1v_2v_3\ldots v_k$ um caminho maximal em $F$ com $k \ge 2$ (Podemos escolher isto, já que $E(F) \ne \varnothing$).
%
Sabemos então, que $v_1$ e $v_k$ são folhas em $F$, senão ou $P$ não seria maximal ou haviria um ciclo em $F$.
%
Considere então $F' := (V(F),E(F)-E(P))$, que é uma floresta (não contém ciclos), com $e(F') < e(F)$, duas folhas a menos ($v_1$ e $v_k$) e $\Delta(F') \ge \Delta(F) - 2$, pois $\forall v\in V(F)\quad d_F(v) \ge d_{F'}(v) - 2$, já que removemos as arestas de um caminho.
%
Por hipótese de indução, temos que $\# folhas(F') \ge \Delta(F')$.
%
Logo, temos o seguinte: 
\begin{eqnarray}
	\#folhas(F) &=& \# folhas(F') + 2  \nonumber \\
		  &\ge& \Delta(F') + 2 	   \nonumber \\
		  &\ge& \Delta(F) - 2 + 2  \nonumber \\
		    &=& \Delta(F) \nonumber
\end{eqnarray}
\fimprova

Como corolário deste resultado temos a resposta da questão, para o caso em que $F$ é conexa.\\

\partQuest{b} Um conjunto ${\cal P} = \{P_1, P_2, \ldots, P_k\}$, onde $P_i$, $1 \le i \le k$, é um caminho. Além disto, os caminhos são disjuntos nos vértices, exceto pelo vertice inicial $x$, que é comum a todos.
%
Por exemplo ${\cal P} = \{ xv_1v_2v_3, xz_1z_2z_3z_4, xq_1q_2\}$, que pode ser visto na Figura \ref{graph:estrelaGrande}.
%
Neste caso, temos $d(x) = k$, $d(y) = 2$ (quando $y$ não está no final de algum caminho) e $d(s) = 1$ para os últimos vértices de cada caminho.
%
Além disto, temos $k$ folhas (uma de cada caminho).
%%%%%%%%%%%%%%%%%%%%%%%%%%%%%%%%%%%%%%%%%%%%%%%%%%%%%
%
%
%                       Delta(G) = #folhas(G)
%
%
%%%%%%%%%%%%%%%%%%%%%%%%%%%%%%%%%%%%%%%%%%%%%%%%%%%%%
\begin{figure} [htb]
        \centering
        \begin{postscript}
                \TinyPicture\VCDraw{%
                \begin{VCPicture}{(0,0)(12,12)}
                        \ChgEdgeArrowStyle{-}
                        %%%%%%%%%%%%%%%%%%%%%%%%%%%% vertices %%%%%%%%%%%%%%%%%%%%%%%%%
                        % Vértice em comum
                        \State[x]{(6,6)}{1}
			% Caminho v
			\State[v_1]{(8,6)}{2} \State[v_2]{(10,6)}{3} \State[v_3]{(12,6)}{4}
                        % Caminho q
                        \State[q_1]{(4,4)}{5} \State[q_2]{(2,2)}{6}
			% Caminho z
			\State[z_1]{(4,8)}{7} \State[z_2]{(2,10)}{8} \State[z_3]{(0,12)}{9}
                        %%%%%%%%%%%%%%%%%%%%%%%%%%%% Arestas %%%%%%%%%%%%%%%%%%%%%%%%%
			% caminho v
                        \EdgeL{1}{2}{} \EdgeL{2}{3}{} \EdgeL{3}{4}{}
			% caminho q
                        \EdgeL{1}{5}{} \EdgeL{5}{6}{}
			% caminho z
                        \EdgeL{1}{7}{} \EdgeR{7}{8}{} \EdgeL{8}{9}{}
                \end{VCPicture}}
        \end{postscript}
        \caption {Grafo $G$ com $\Delta(G) = \#folhas(G)$.}
	\label{graph:estrelaGrande}
\end{figure}



\quest{4.1.2}

Vamos provar que a) $\leftrightarrow$ c) e b) $\leftrightarrow$ c).

\partQuest{a}

\partQuest{b}
($\Rightarrow$) Seja $G$ uma floresta com $n-1$ arestas.
%
Como florestas são acíclicas, basta mostrar que $c(G) = 1$ e teremos então uma árvore. 
%
Considere então ${\cal C} = \{c_1, c_2, \ldots, c_k\}$ as componentes conexas de $G$.
%
Cada componente $c_i$ é um grafo conexo, acíclico e com $e(C_i) = v(c_i) - 1$, pois é uma árvore ({\bf Teorema 4.3}).
%
Desta forma, a quantidade de arestas em $G$ é dada por:
\begin{eqnarray}
	e(G) &=& \sum_{c_i \in {\cal C}} e(c_i)      \nonumber \\
	     &=& \sum_{c_i \in {\cal C}} v(c_i) - 1  \nonumber \\
	     &=& \sum_{c_i \in {\cal C}} v(c_i) - \sum_{c_i \in {\cal C}} 1 \nonumber \\
	     &=& n - k \label{eq:componentes}
\end{eqnarray}

Logo, como temos $n-1$ arestas em $G$, então $k = 1$, com o conjunto $|{\cal C}| = 1$ e, portanto, $G$ uma árvore.

($\Leftarrow$) Seja $G$ uma árvore, então, por definição $G$ é floresta. Além disto, $c(G) = 1$ e, utilizando a equação \ref{eq:componentes}, temos que $e(G) = n - 1$ e o resultado segue.
\fimprova


\quest{4.1.6}

Seja $x \in V(D)$ e $X=\{v: v \in v(D)$ e $\exists xDv\}$.
%
Considere então o subgrafo induzido $D'$ pelos vértices em $X$ ($D' := D[X]$).
%
Remova todos os arcos com cabeça em $x$, pois queremos um $x$-{\it branching}.
%
Todos os vértices ainda permanecem alcançáveis, pois estes arcos não estavam em nenhum caminho orientado entre $x$ e $v$, $v \in V(D')$ (estavam em passeios orientados).
%
Se $\forall v \in V(D')\quad d^-(v) = 1$, então já temos um $x$-{\it branching}.
%
Caso contrário, seja $v \in V(D')$, onde $d^-(v) > 1$, então remova um dos arcos com cabeça em $v$.
%
Neste caso, $v$ ainda será alcançável a partir de $x$, pois ainda há um caminho $xD'v$ utilizando um dos arcos $(v',v)$ restantes, pois $\exists xD'v'$.
%
Desta forma, remover um arco de vértices $v$, onde $d^-(v) > 1$, preserva a existência de caminho entre $x$ e $v$.
%
Portanto, repetindo este processo, eventualmente obteremos um $x$-{\it branching}, já que $e(D')$ é finito.
\fimprova

\quest{4.1.20}

\partQuest{a}
($\rightarrow$) Suponha por absurdo que $T_1 \cap T_2 = \emptyset$. Se
$u \in V(T_1)$ e $v \in V(T_2)$ então não existe caminho de $u$ para $v$ em
$T_1 \cup T_2$. Mas isso contraria a hipótese de que $T_1 \cup T_2$ é uma
subárvore de $T$.

\partQuest{b}
Prova por indução forte em $|T_f| = n$.

Hipótese: Se $T_f$ é uma família de subárvores de uma árvore $T$ com $|T_f| = k$,
$ 1 \le k \le n$, se quaisquer 2 membros de $T_f$ possuem um vértice em comum,
então há um vértice de $T$ que pertence a todos os membros de $T_f$.

Passo: Seja $T_f$ uma família de subárvores da árvore $T$ como na hipótese e tal
que $|T_f| = n + 1$. Sejam $T'$ e $T''$ subárvores de $T_f$ tal que
$T' \cap T'' \ne \emptyset$. Sabemos de (a) que $T' \cap T''$ é subárvore de $T$.
Seja $T'_f = T_f - \{T', T''\} \cup \{T' \cap T''\}$ a família obtida por
substituir $T'$ e $T''$ por sua intersecção em $T_f$. Em $T'$, temos que
$T_i \cap T_j \ne \emptyset$, $\forall T_i, T_j$. Então, por H.I. existe um
vértice $v$ de $T$ que pertence a todos os membros de $T'_f$. Em particular, $v$
pertence a $T' \cap T''$ e, portanto, pertence a $T'$ e $T''$.
\fimprova

\quest{4.2.1}

\partQuest{a}


\partQuest{b} Seja $G$ um grafo conexo qualquer e $e = vx \in E(G)$, temos que para cada subárvore geradora $T$ de $G$, ou $e \in E(T)$ ou $e \notin E(T)$.
%
Se $e \notin E(T)$, então $T$ é arvore geradora de $G\backslash e$, caso contrário, se $e \in E(T)$, então pela letra (a) temos que $T/e$ é árvore geradora de $G/e$.
%
Logo, $$t(G) = t(G\backslash e) + t(G/e)$$

\quest{4.2.3}
\partQuest{a}
Vou mostrar como construir a {\it distance Tree} $T$ e depois demonstrarei que $\forall v \in V(G)\quad d_T(x,v) = d_G(x,v)$

Considere inicialmente $T:=({x},\varnothing)$, uma árvore que será aumentada a cada passo do processo. Seja  $V' = \{v: v \in V(G) - V(T)$ e $uv \in \partial(V(T))\}$.
%
Faça $T' := (V(T)\cup V', E(T)\cup E')$,  onde $E' \subseteq \partial(V(T))$ é o conjunto das arestas com extremos não repetidos em $V(G) - V(T)$ (Se $k$ arestas tem o mesmo extremo em $V(G) - V(T)$, escolha qualquer uma delas para fazer parte de $E'$).
%
Então, repita o processo com $T'$ enquanto $V(T') \ne V(G)$.

A cada passo inserimos, pelo menos, um vértice e uma aresta em $T$, pois $\partial(V) \ne \varnothing$ para todo $V \subset V(G)$ pois $V$ é conexo.
%
Portanto, eventualmente, o processo alcança $V(T) = V(G)$, logo o subgrafo $T$ de $G$ é gerador e, além disto, é uma árvore pois é conexo (pela forma como foi construído) e acíclico, pois as arestas inseridas em $T$ não geram ciclos, pois criam o primeiro e único caminho entre $u$ e $x$, onde $u \in V(T)$ e $v \in V(G)-V(T)$.

Mostrarei agora que $\forall v \in V(G)\quad d_T(x,v) = d_G(x,v)$. A prova será por indução fraca em $d_G(x,v)$.
%
Na base, considere que $d_G(x,v) = 1$, logo, pela forma com que $T$ é construída, temos que na primeira iteração do processo $T=({x},\varnothing)$, e $v\in V'$ pois ${x,v} \in \partial(\{x\})$.
%
Logo, pelo menos uma aresta incidente em $v$ será adicionada a $T$ e, portanto $d_T(x,v) = 1$.
%
Como Hipótese de Indução, tenho que se $d_G(x,v) = k$ então $d_T(x,v)$.
%
No passo, considere $v' \in V(G)$, onde, $d_G(x,v') = k + 1$.
%
Seja então $P_i = xv_{1i}v_{2i}\ldots v_{ji}v'$ caminhos de $x$ a $v'$ em $G$ com comprimento $k + 1$. 
%
Sabemos que $d_G(x,v_{ji}) = k$ $\forall i$, senão haveria um caminho menor que todos os caminhos $P_i$ de $x$ a $v'$.
%
Logo, por Hipótese de Indução, temos que $d_T(x,v_{ji}) = k$ e portanto, uma das arestas ${v_{ji},v'}$ será adicionada a $T$ em algum passo do processo descrito acima.
%
Portanto, $d_T(x,v') = d_T(x,v_{ji}) + 1 = k + 1$.
\fimprova

\partQuest{b}


\quest{4.2.9}

\partQuest{a}
($\Leftarrow$) Primeiramente vou mostrar como construir o {\it spanning x-branching} $S$ de $D$ para o $x \in V(D)$ como dado na hipótese.
%
Considere inicialmente $S := (\{x\}, \varnothing)$.
%
Enquanto $V(D)-V(S) \ne \varnothing$ repita o seguinte procedimento.
%
Como $V(D)-V(S) \ne \varnothing$, então existe, no mínimo, um arco $(v,v')$, onde $v \in V(D)$ e $v' \in V(S)$, pois $\partial^+(S) \ne \varnothing$.
%
Seja $V' = \{v: (u,v) \in \partial^+(S)$ e $E' \subseteq \partial^+(S)$ um subconjunto com apenas um arco incidindo em cada vértice (se mais de um arco incide em $z \in V(D)-V(S)$, remova uma delas).
%
Atualize $S := (V(S)\cup V', A(S) \cup E')$.

Basta-nos demonstrar que $S$ gerado é um {\it spanning x-branching}.
%
É gerador pois, a cada iteração incorpora, no mínimo, um vértice $v \in V(D)-V(S)$ e $V(D)$ é finito.
%
É {\it x-branching} pois $d^-(v) = 0$, pois $v$ nunca é adicionado novamente ao grafo $S$, logo $\nexists (.,x) \in A(S)$.
%
Além disto, $d^-(v) = 1$, $v\ne x$, pois a partir do momento em que um vértice $v$ é adicionado ao conjunto $V(S)$, ele nunca mais o será, pois não pertencerá a $V(D)-V(S)$.

($\Rightarrow$) Seja $S$ um {\it spanning x-branching} de $D$.
%
Seja $X \subset V(D)$, onde $x \in X$. Como $S$ é {\it spanning x-branching}, temos que $d^-(v) = 1$, $\forall v \in V(D)-X$ e, por definição, não há ciclos em $S$.
%
Logo, seja $P = v_1v_2\ldots v_k$ um caminho maximal em $V(D) - X$. Sabemos que $d^-(v_1) = 1$ e este arco incidente não pode ter cauda em $V(D) - X$ senão $P$ não seria maximal, ou existiria um ciclo em $S$ e em ambos os casos teríamos um absurdo.
%
Logo, $\exists (v',v_1) \in A(S)$, onde $v' \in X$, logo $|\partial^+(X)| \ge 1$, e, portanto, $\partial^+(X) \ne \varnothing$.
\fimprova

\partQuest{b} ($\Leftarrow$) Seja $D$ um digrafo para o qual existe um {\it spanning v-branching} $\forall v \in V(D)$.
%
Considere $x,y \in V(D)$, então num {\it spanning x-branching} existe um caminho orientado de $x$ para $y$, analogamente, num {\it spanning y-branching} há um caminho orientado de $y$ para $x$.
%
Desta forma, $D$ é fortemente conexo.

($\Rightarrow$) Seja $D$ um digrafo fortemente conexo.
%
Seja $x \in V(D)$ um vértice qualquer, temos que $\forall X \subset V(D)$, onde $x \in X$, existe um caminho de $x$ para $y \in V(D) - X$. Logo, $\partial^+(X) \ne \varnothing$ e, portanto, pela letra {\bf a)} temos que existe um {\it spanning x-branching} em $D$.


\quest{4.3.2}

\partQuest{a}
Seja $G$ um grafo qualquer e $T_1$ e $T_2$ duas subárvores geradoras de $G$.
%
Seja $e = xy \in T_1\backslash T_2$, considere então $T_1 - e$, e sabemos que como $e$ é uma aresta de corte, geraremos duas componentes conexas $X$ e $Y$, com os vértices alcancáveis a partir de $x$ e $y$, respectivamente (Veja a Figura \ref{graph:componentesT1-e}).
%
Como $e \notin T_2$, então $T_2 + e$ contém um ciclo fundamental e, portanto, existe uma aresta $f$ que liga algum $x' \in X$ a um $y' \in Y$.

Basta então mostrar que $T_1' := \{T_1 - e\} + f$, é acíclico e conexo.
%
Primeiramente, $T_1'$ é conexo pois existe caminho entre qualquer parte de vértices, $v$ e $v'$, internos em $X$ e $Y$ e, além disto, se $v$ e $v'$ estiverem em conjuntos diferentes ($v \in X$ e $v' \in Y$, sem perda de generalidade), então $\exists vXx'y'Yv'$ um caminho em $T_1'$.
%
Alem disto, $T_1'$ é acíclico, pois $T_1 - e$ o é, e se $f$ gerasse um ciclo em $T_1 - e$, então $\exists x'{T_1 - e}y'$ que não usa $f$, o que é uma contradição pois $x' \in X$ e $y' \in Y$ duas componentes conexas distintas em $T_1 - e$.
%%%%%%%%%%%%%%%%%%%%%%%%%%%%%%%%%%%%%%%%%%%%%%%%%%%%%
%
%
%                       Delta(G) = #folhas(G)
%
%
%%%%%%%%%%%%%%%%%%%%%%%%%%%%%%%%%%%%%%%%%%%%%%%%%%%%%
\begin{figure} [htb]
        \centering
        \begin{postscript}
                \TinyPicture\VCDraw{%
                \begin{VCPicture}{(0,0)(12,12)}
                        \ChgEdgeArrowStyle{-}
			%%%%%%%%%%%%%%%%%%%%%%%%%%%% Componentes %%%%%%%%%%%%%%%%%%%%%
			\FixStateDiameter{6cm} \ChgStateLabelScale{3} 
			\StateVar[X]{(2,6)}{X} \State[Y]{(10,6)}{X}  
			\LargeState \RstStateLabelScale 
                        %%%%%%%%%%%%%%%%%%%%%%%%%%%% vertices %%%%%%%%%%%%%%%%%%%%%%%%%
                        % Vértices de X
                        \State[x']{(4,4)}{1} \State[x]{(4,8)}{2}
                        % Vértices de y
                        \State[y']{(8,4)}{3} \State[y]{(8,8)}{4}
                        %%%%%%%%%%%%%%%%%%%%%%%%%%%% Arestas %%%%%%%%%%%%%%%%%%%%%%%%%
                        % arestas e e f
                        \ArcR{1}{3}{f} 
			\ChgEdgeLineStyle{dashed}
			\ArcL{2}{4}{e}
                \end{VCPicture}}
        \end{postscript}
        \caption {O grafo $T_1 - e$ e as arestas $e$ e $f$.}
        \label{graph:}
\end{figure}
\fimprova

\partQuest{b} É exatamente a questão anterior, invertendo $f$ por $e$.
\fimprova


\quest{4.3.10}

\partQuest{a} Seja $G$ um grafo qualquer e $T_1$ e $T_2$ árvores geradoras de $G$ disjuntas. Vamos construir uma família de ciclos para depois utilizarmos o corolário {\bf 2.16} e mostrar que existe um subgrafo gerador par de $G$, logo, euleriano.

Considere $\overline{T_1}$, a {\it cotree} de $T_1$.
%
Seja ${\cal C} = \{c_e:$ um ciclo fundamental com $e \in \overline{T_1}\}$ uma família de cíclos.
%
Note que como $T_2 \subseteq \overline{T_1}$, então $\forall v \in V(G)\quad \exists c_e \in {\cal C}$, tal que $v \in c_e$, em especial quando $e = vx$, para algum $x \in V(G)$.

Portanto, o grafo $H_1 = \bigtriangleup {\cal C}$ é gerador e, pelo corolário {\bf 2.16}, é também um grafo par.
%
Logo, pelo Teorema {\bf 3.5} $G$ é euleriano.
\fimprova

\partQuest{b} Considere o grafo $H_1$ como gerado na letra {\bf a)} e, de forma semelhante, o grafo $H_2$ baseado em $T_2$ e $\overline{T_2}$.
%
Pelo que foi demonstrado em {\bf a)} temos que $H_1$ e $H_2$ são subgrafos pares geradores de $G$.%
Além disto, pelo Teorema {\bf 4.10} temos que $H_i \cap \overline{T_i} = \overline{T_i}$, $i = 1,2$, logo $\overline{T_i} \in H_i$.
%
Note também que $T_2 \subseteq \overline{T_1}$ e $T_1 \subseteq \overline{T_2}$, portanto: 
\begin{eqnarray}
	H_1 \cap H_2 &=& \overline{T_1} \cup \{H_1 - \overline{T_1}\} \cup \overline{T_2} \cup \{H_2 - \overline{T_2}\} \nonumber \\
		     &=& G \nonumber
\end{eqnarray}
\fimprova


% Exemplos
%\quest{???} Exemplos de grafos$\ldots$

\vspace{0.2cm}


%%%%%%%%%%%%%%%%%%%%%%%%%%%%%%%%%%%%%%%%%%%%%%%%%%%%%
%
%
%			Petersen
%
%
%%%%%%%%%%%%%%%%%%%%%%%%%%%%%%%%%%%%%%%%%%%%%%%%%%%%%
\begin{figure} [htb]
	\centering
	\begin{postscript}
		\TinyPicture\VCDraw{%
		\begin{VCPicture}{(0,0)(10,9)}
			\ChgEdgeArrowStyle{-}
			%%%%%%%%%%%%%%%%%%%%%%%%%%%% vertices %%%%%%%%%%%%%%%%%%%%%%%%%
			% Anel externo
			\State[v_1]{(5,9)}{1} \State[v_2]{(10,5)}{2} \State[v_3]{(8,0)}{3} \State[v_4]{(2,0)}{4} \State[v_5]{(0,5)}{5} 
			% Anel interno
			\State[v_6]{(5,7)}{6} \State[v_7]{(7,5)}{7} \State[v_8]{(6,2)}{8} \State[v_9]{(4,2)}{9} \State[v_{10}]{(3,5)}{10}
			%%%%%%%%%%%%%%%%%%%%%%%%%%%% Arestas %%%%%%%%%%%%%%%%%%%%%%%%%			
			\EdgeL{1}{2}{} \EdgeL{2}{3}{} \EdgeL{3}{4}{} \EdgeL{4}{5}{} \EdgeL{5}{1}{}
			\EdgeL{1}{6}{} \EdgeL{2}{7}{} \EdgeL{3}{8}{} \EdgeL{4}{9}{} \EdgeL{5}{10}{}
			\EdgeL{6}{8}{} \EdgeR[.1]{6}{9}{} \EdgeL{7}{9}{} \EdgeR{7}{10}{} \EdgeL{8}{10}{}
			%
		\end{VCPicture}}
	\end{postscript}
	\caption {Petersen Graph.}
\end{figure}


%%%%%%%%%%%%%%%%%%%%%%%%%%%%%%%%%%%%%%%%%%%%%%%%%%%%%
%
%
%		O "NOSSO AMIGO"???
%
%
%%%%%%%%%%%%%%%%%%%%%%%%%%%%%%%%%%%%%%%%%%%%%%%%%%%%%
\begin{figure} [htb]
	\centering
	\begin{postscript}
		\TinyPicture\VCDraw{%
		\begin{VCPicture}{(0,-2)(6,2)}
			\ChgEdgeArrowStyle{-}
			%%%%%%%%%%%%%%%%%%%%%%%%%%%% vertices %%%%%%%%%%%%%%%%%%%%%%%%%
			\State[v_1]{(0,6)}{1} \State[v_2]{(10,6)}{2} \State[v_3]{(3,3)}{3} \State[v_4]{(7,3)}{4} \State[v_5]{(0,0)}{5} \State[v_6]{(10,0)}{6}
			% Edges
			\EdgeL{1}{2}{e_1} \EdgeL{1}{3}{e_2} \EdgeL{1}{5}{e_3} \EdgeL{3}{5}{e_4} \EdgeL{3}{4}{e_5}
			\EdgeL{5}{6}{e_6} \EdgeL{2}{4}{e_7} \EdgeL{4}{6}{e_8} \EdgeL{2}{6}{e_9}
			%
		\end{VCPicture}}
	\end{postscript}
	\caption {O Nosso amigo $\overline{C_6}$.}
\end{figure}


\end{document}



